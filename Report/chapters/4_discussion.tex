\chapter{Conclusion}\label{sec:conclusion}

Using the DVS for tracking has some clear advantages as the high sampling speed enables the tracking of LEDs pulsed with frequencies above a 1000 Hz. Compared to normal high speed cameras, the data output and thus the processing is reduced, as only change is advertised by the camera. This makes the DVS also interesting for embedded processing.
The measurements revealed that the DVS is able to reacquire stable tracking after as the quadrotor flips with negligible delay as soon as the LEDs are visible again. In comparison to the PTAM used by the quadrotor, it is more than twice as fast. The DVS thus has a clear advantage in comparison to conventional cameras as it does not suffer from blurring. Nevertheless, we used active markers in our experiment which are easier to identify than image features. Due to the fact that the DVS not only needs motion in order to produce input but also has a lower resolution than conventional imaging sensors it is not clear yet, how well features from the environment can be used as a tracking reference. The pose estimation has proven to be more accurate than PTAM and should thus perform well in helicopter navigation. Nevertheless, the low resolution of the DVS limits the range in which a helicopter can be tracked robustly.
As there are possible improvements to our algorithm in terms of stability and robustness, we reckon that is should be possible to even improve the pose estimation accuracy on the DVS. A possible approach could include considering not only local maxima from single pixels, but averaging over several pixels activated by an LED. Alternatively another particle filter could be used to smoothen the position readings over time. This approach would also help with the occasional appearance of misdetections.
On the hardware side we found a couple of improvements as well. For further experiments reliable drone control is important. While reducing or better balancing the payload could have an impact, we were also not sure if the highly used rotor blades or other parts of the quadcopter were responsible for the unstable flight. As recalibration of the internals seems unfeasible one might also consider using a different drone in the future. To improve visibility of the markers, LEDs with higher angle and power output would be beneficial. This requires some additional hardware though which needs to be considered again in terms of payload. \\

Apart from a number of possible improvements, the approach has shown to be feasible and has demonstrated the advantages of using and event-driven approach for vision based robotics. As this utilization of asynchronous vision is a rather novel approach in robotics it would be interesting to use this camera in other navigational tasks in future projects, for example for visual SLAM on autonomous ground vehicles. While small image features as used in SIFT \cite{SIFT} might be difficult to use due to resolution line feature extraction, as describe in \cite{LineTracking}, could be a feasible approach. Additionally, as edges are the  natural source for DVS events it would have an advantage in terms of processing compared to normal cameras.