\chapter{Introduction}\label{sec:introduction}

A crucial property of micro aerial vehicle (MAV) navigation and control is the estimation of their position towards a certain reference in their environment. There exist several methods to achieve this, the most popular one being the use of cameras. The cameras can be either placed on the MAV itself or statically in the environment to track MAVs in their field of view. In order to estimate the pose, specific features in the camera images must be found and tracked. This works well as long as there are significant features to be found in the image. Since conventional camera sensors generally provide not more than 30 frames per second, fast motion will blur the camera image and thus hinder feature detection.  Therefore, during an aggressive maneuver tracking is usually lost and the tracking needs to be reacquired after getting clear images again. To re-establish a lost track a certain amount of data needs to be gathered which induces a significant delay in the reacquisition of tracking. To overcome this problem one could of course employ cameras with higher frame rates. This would come at an expense though, not only in terms of data processing but also in much higher price. Therefore we wanted to find a solution which is not only computationally cheaper while still providing high speed but also comes at a lower price as today’s high speed cameras.
The Institute of Neuroinformatics at the University of Zurich and ETH has developed a dynamic vision sensor (DVS). Inspired by the human retina, each pixel fires asynchronously and only if there is change in illumination on this specific pixel. Thus, the data from the DVS are not frames but events depicting the time and place of an impulse. The change of illumination is thereby binary, i.e. only the direction of change is advertised (This binary representation in events will be referred to as polarity in the following). That provides high sampling rates, while maintaining a relatively low data output. As the processing is done asynchronously the sampling rate is not fixed but rather dependent on the amount of overall activity on the pixels. There are several reasons for that, for example limit set by the serial time-stamping mechanism, as well as the data output via USB 2.0. Nevertheless, the camera provides up to 1 million events per second. Using such special hardware comes at an expense though as the camera has a comparably low resolution of 128x128 pixels. For a deeper insight into the camera technology please refer to [DVS paper]. 
In order to facilitate tracking, given the high temporal resolution of the DVS, we decided to use active markers in the form of differently pulsed LEDs with the DVS. The feasibility and high robustness of this approach had already been demonstrated by [Matthias]. The LEDs were attached to a quadrotor helicopter, while the quadrotor itself was tracked by the DVS.  For the marker tracking we developed an algorithm able to track several LEDs pulsed with different frequencies while using an already existing library for pose estimation. In the following part the challenges an preparations are discussed. After that a detailed description of the algorithm will be given followed by the experiments and results. Finally a conclusion summarizes our findings as well as discusses possible improvements. Also, an outlook on further interesting projects will be given.


\section{Related Work}\label{sec:related_work}
