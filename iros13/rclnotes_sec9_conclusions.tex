\vfill\newpage


\section{Conclusions\label{sec:conclusion}}

Fast robots need fast sensors. A \emph{dynamic vision sensor} (DVS)
returns changes in the visual field with a latency of a few microseconds.
This technology is the most promising candidate for enabling highly
aggressive autonomous maneuvers for flying robots. The current prototypes
suffer a few limitations, such as a relatively low resolution, which
are being worked upon. In the mean time, the sensing pipeline must
be completely re-designed to take advantage of the low latency.

This paper has presented the first pose tracking application using
DVS data. We have shown that the DVS can detect Active LEDs Markers
(\ALMs) and disambiguate their identity if different blinking frequencies
are used. The algorithm that the we developed uses a Bayesian framework,
in which we accumulate evidence of every single event into ``evidence
maps'' that are tuned to a particular frequency. The temporal interval
can be tuned and it is a tradeoff between latency and precision. In
our experimental conditions it was possible to have a latency of only
1~ms. After detection, we used a particle filter and a multi-hypothesis
tracker. 

We have evaluated the use of this technology for tracking the motion
of a quadrotor during an aggressive maneuver. Experiments show that
the DVS is able to reacquire stable tracking with negligible delay
as soon as the LEDs are visible again, without suffering from motion
blur, which limits the traditional CMOS-based conventional feature
tracking solution. However, the precision in reconstructing the pose
is limited because of the low sensor resolution. Future work involving
the hardware include improving the \ALMs by increasing their power
and their angular emittance field, as we have found these to be the
main limitations.

In conclusion, DVS-based \ALM tracking promises to be a feasible
technology that can be used for fast tracking in robotics.







\medskip

\noindent \textbf{Acknowledgements}: We gratefully acknowledge the
contribution of Christian Saner for helping with the ARDrone software
development, the assistance of Yves Albers-Sch\"onberg as our pilot
during test flights and in recording ROS logs, and the assistance
of Matia Pizzoli in the CMOS-based tracking experiments.
